%% LyX 2.1.4 created this file.  For more info, see http://www.lyx.org/.
%% Do not edit unless you really know what you are doing.
\documentclass[a4paper,english]{article}
\usepackage[T1]{fontenc}
\usepackage[latin9]{inputenc}
\usepackage{babel}
\usepackage{amsmath}
\usepackage{amsthm}
\usepackage{amssymb}
\usepackage{esint}
\usepackage[authoryear]{natbib}
\usepackage[unicode=true,pdfusetitle,
 bookmarks=true,bookmarksnumbered=false,bookmarksopen=false,
 breaklinks=false,pdfborder={0 0 1},backref=false,colorlinks=false]
 {hyperref}

\makeatletter

%%%%%%%%%%%%%%%%%%%%%%%%%%%%%% LyX specific LaTeX commands.
\pdfpageheight\paperheight
\pdfpagewidth\paperwidth


%%%%%%%%%%%%%%%%%%%%%%%%%%%%%% User specified LaTeX commands.
\usepackage{amsmath}
\usepackage{amsfonts}
\usepackage{amsthm}
\usepackage{enumerate}
\usepackage{geometry}

%\usepackage{index}
%\usepackage{mdwlist}
\usepackage{natbib}
%\usepackage{microtype}
\usepackage{soul}

\usepackage{url}


\newcommand{\diag}{\mathop{\mathrm{diag}}}
\newcommand{\trace}{\mathop{\mathrm{tr}}}
\newcommand{\median}{\mathop{\mathrm{median}}}

\let\oldenumerate=\enumerate
\def\enumerate{
\oldenumerate
\setlength{\itemsep}{0pt}
}
\let\olditemize=\itemize
\def\itemize{
\olditemize
\setlength{\itemsep}{0pt}
}

\newcommand{\postmeta}[1]{}

\makeatother

\begin{document}

\title{ML/Math/Stats Summary}


\author{Wittawat Jitkrittum}

\maketitle
\tableofcontents{}


\section{Probability $p(\|x\|_{2}<R)$ where $x\sim\mathcal{N}(0,\Sigma)$
and $x\in\mathbb{R}^{D}$}

\postmeta{
Title: Probability $p(\|x\|_{2}<R)$ where $x\sim\mathcal{N}(0,\Sigma)$  and $x\in\mathbb{R}^{D}$
Date: 2014-06-29 
Tags: probability, statistics
Slug: prob_gaussian_ball
}

A friend of mine recently asked ``What is the value of $p(\|x\|_{2}<R)$
where $x\sim\mathcal{N}(0,\Sigma)$ and $x\in\mathbb{R}^{D}$?''
The question is not as easy as it looks. Although I did not get the
answer, I arrived at some expression which is close to the goal. Before
I show what I did, here is one result we will need.
\begin{itemize}
\item If $z^{2}\sim\chi^{2}(D)$ and $v$ is a positive constant, then $vz^{2}\sim\mathrm{Gamma}(\frac{D}{2},2v)$
(shape-scale parameterization). 
\end{itemize}
Here is what I did. Eigen-decompose $\Sigma=UVU^{\top}$. Note that
$p(\|x\|_{2}<R)=p(x^{\top}x<R^{2})$. The goal here is to find the
CDF of $x^{\top}x$. Let $y:=U^{\top}x$. It follows that $x^{\top}x=y^{\top}y=\sum_{i=1}^{D}y_{i}^{2}$
because $U$ is an orthogonal matrix. So, $p(x^{\top}x<R^{2})=p(y^{\top}y<R^{2})$
. From the property of the normal distribution, we know that $y=U^{\top}x\sim\mathcal{N}(0,V)$.
Equivalently, $y_{i}\sim\mathcal{N}(0,v_{i})$ where $v_{i}:=V_{ii}$.
Let $z\sim\mathcal{N}(0,1)$ be the standard normal random variable.
We can rewrite $y_{i}$ as $y_{i}=\sqrt{v_{i}}z$, and so $y_{i}^{2}=v_{i}z^{2}$,
where $z^{2}\sim\chi^{2}(1)$, the Chi-squared distribution with one
degree of freedom. From the previous result, we have $y_{i}^{2}\sim\mathrm{Gamma}(\frac{1}{2},2v_{i})$.
This implies that the CDF of $y^{\top}y$ is the same as the CDF of
a sum of independent Gamma random variables. According to \href{https://projecteuclid.org/euclid.ejs/1403812157}{Covo and Elalouf, 2014}
and \href{https://www.researchgate.net/publication/225242960_The_Distribution_of_the_Sum_of_Independent_Gamma_Random_Variables}{Moschopoulos, 1985},
the CDF is not trivial to compute. 


\section{Learning the parameters of determinantal point process kernels \citep{Affandi2014}}

22 June 2015. A determinantal point process (DPP) is a probability
measure on the $2^{N}$ possible subsets $A$ of a discrete set $\Omega=\{x_{1},\ldots,x_{N}\}$:
\[
P_{L}(A)=\frac{\det(L_{A})}{\det(L+I)}
\]
where $L_{A}$ is a submatrix of the $N\times N$ matrix $L$ indexed
by the members in the set $A$. Given a set of $T$ samples $S=\{A^{1},\ldots,A^{T}\}$,
the parameters $\theta$ of the kernel $L$ can be learned by performing
a gradient ascent on the log likelihood $\mathcal{L}(\theta)$:
\[
\mathcal{L}(\theta)=\sum_{t=1}^{T}\log\det(L_{A^{t}})-T\log\det(L(\theta)+I).
\]
When $N$ is large, however, computing the likelihood or the derivative
is inefficient. A Bayesian learning of DPPs relies on sampling techniqes
such as Metropolis-Hastings (MG) and slice sampling. In MH, we use
a proposal distribution $f(\hat{\theta}|\theta_{i})$ to generate
a candidate $\hat{\theta}$ given the current parameters $\theta_{i}$
which are then accepted or rejected with probability $\min(1,r)$
where 
\[
r=\left(\frac{p\left(\hat{\theta}|S\right)f(\theta_{i}|\hat{\theta})}{p(\theta_{i}|S)f(\hat{\theta}|\theta_{i})}\right).
\]
Notice that the normalizer $\det(L+I)$ depends on $\theta$ and does
not cancel out in the ratio in $r$. In fact, evaluation of the normalizer\footnote{Here we consider $\Omega$ to be an uncountable set; hence, the operator
$L$ has infinitely many eigenvalues.} $\det(L(\theta)+I)=\prod_{n=1}^{\infty}(\lambda_{n}(\theta)+1)$
is difficult. The main idea of \citet{Affandi2014} is to use lower
and upper bounds of $p(\hat{\theta}|S)$ in place of $p(\hat{\theta}|S)$
in the MH acceptance ratio. 
\begin{align*}
\text{lower bound: }\prod_{n=1}^{M}(1+\lambda_{n}) & \leq\prod_{n=1}^{\infty}(1+\lambda_{n}),\\
\text{upper bound: }\prod_{n=1}^{\infty}(1+\lambda_{n}) & \leq\exp\left\{ \trace(L)-\sum_{n=1}^{M}\lambda_{n}\right\} \left[\prod_{n=1}^{M}(1+\lambda_{n})\right],
\end{align*}
where $\trace(L)$ can be computed as either $\sum_{i=1}^{N}L_{ii}$
in the discrete case or $\int_{\Omega}L(x,x)\,\mathrm{d}x$ in the
continuous case. In each MH step, two quantities $r^{+}$ and $r^{-}$
are computed:
\[
r^{+}=\left(\frac{p^{+}\left(\hat{\theta}|S\right)f(\theta_{i}|\hat{\theta})}{p^{-}(\theta_{i}|S)f(\hat{\theta}|\theta_{i})}\right),\thinspace\thinspace\thinspace\thinspace\thinspace\thinspace\thinspace\thinspace r^{-}=\left(\frac{p^{-}\left(\hat{\theta}|S\right)f(\theta_{i}|\hat{\theta})}{p^{+}(\theta_{i}|S)f(\hat{\theta}|\theta_{i})}\right).
\]
If $u<\min(1,r^{-1})$ where $u\sim\text{Uniform}[0,1]$, the immediately
reject because it follows that $u<\min(1,r)$ as well. Similarly,
if $u>\min(1,r^{+})$, immediately accept the proposal. If $u\in(r^{-},r^{+})$,
then tighten the bounds until a decision can be made. This idea is
also applicable to the slice sampling. 

It is unclear if this scheme will work in the case that $q_{u}$ is
high dimensional. 


\section{Variational Holder bound}

This post summarizes 

\begin{verbatim}
Approximate Inference with the Variational Holder Bound
Guillaume Bouchard, Balaji Lakshminarayanan
arXiv:1506.06100, 2015.
\end{verbatim}

\href{http://arxiv.org/abs/1506.06100}{The paper}proposes the variational
H�lder bound as an alternative to the variational Bayes (VB) for approximante
Bayesian inference. The idea is to construct an \emph{upper bound}
(as opposed to a lower bound as in VB) to the marginal likelihood,
and \emph{minimize} it with respect to an introduced pivot function
$\Psi$. Let $x$ represent the observed samples and $z$ represent
latent variables. The bound $\overline{I}_{\alpha}$ is constructed
as
\begin{align*}
I^{*} & =\int p(z)p(x|z)\,\mathrm{d}z=p(x)\\
 & =\int p(z)\Psi p(x|z)/\Psi\,\mathrm{d}z\\
\text{(H�lder's inequality)} & \leq\left(\int\left(p(z)\Psi\right)^{\alpha_{1}}\,\mathrm{d}z\right)^{1/\alpha_{1}}\left(\int\left(p(x|z)/\Psi\right)^{\alpha_{2}}\,\mathrm{d}z\right)^{1/\alpha_{2}}\\
 & :=\overline{I}_{\alpha}
\end{align*}
where $\frac{1}{\alpha_{1}}+\frac{1}{\alpha_{2}}=1$. With an optimized
$\Psi$, it turns out that $p(z|x)$ can be approximated well with
either $\frac{p(z)\Psi}{\int(p(z)\Psi)^{\alpha_{1}}\,\mathrm{d}z}$
or $\frac{p(x|z)/\Psi}{\int(p(x|z)/\Psi)^{\alpha_{2}}\,\mathrm{d}z}$
depending on the chosen value of $\alpha_{1}$.

\bibliographystyle{plainnat}
\bibliography{ml_learn}

\end{document}
